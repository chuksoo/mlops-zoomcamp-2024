\documentclass[letterpaper,12pt,notitlepage,twoside]{report}

\newcommand{\watermark}{hello there}
\newcommand{\booktopic}{A Topic}
\newcommand{\lastname}{An Author}
\usepackage{NoteStyle}

\title{A Nice Title}
\author{Me, Myself, and I}
\date{\footnotesize Last Updated: \today}

\hypersetup{
  pdftitle={A Topic},
  pdfauthor={Me, Myself, and I},
  pdfsubject={},
}

% ToC on the title page:
% https://tex.stackexchange.com/a/45863
\makeatletter
\newcommand*{\toccontents}{\@starttoc{toc}}
\makeatother

\begin{document}

\begin{titlepage}
  \pagestyle{plain}
  \maketitle
  \setcounter{tocdepth}{2}

  \pdfbookmark[section]{\contentsname}{toc}
  \toccontents
  % Separate page:
  % \tableofcontents
\end{titlepage}

\chapter{Welcome} \label{ch:1}
\inspiration{This is a quote from someone}{Author}{Optional Source}

\firstletter{Hello there}, and welcome to \booktopic. This is literally just an
example with random content that demonstrates how the template will look when
used in book form.


\section{On Colorful Boxes}
There are lots of different boxes you can make:
%
\begin{itemize}
  \item Like this one,
    \begin{notebox}
      which is wrapped in gray. I use it for notes.\ldots
    \end{notebox}

  \item Or this one,
    \begin{funfact}
      which is wrapped in red. I use it for fun facts or other asides\ldots
    \end{funfact}

  \item Or this one,
    \begin{mathaside}
      which is wrapped in blue and used for mathy stuff.
    \end{mathaside}

  \item Or this last one,
    \begin{example}
      which is wrapped in green. With a title, it's used for enumerated examples
      (see \smtt{\textbackslash extitle} and \smtt{\textbackslash excounter}).
      Observe:
    \end{example}

    \begin{example}[frametitle=\extitle{Test}]
      This is an example. What's the answer to $2+2$?
      \answer{Obviously 4, lol.}
    \end{example}

    \begin{example}[frametitle=\extitle{Test Again}]
      This one will increment the counter automatically, resetting for each
      chapter.
    \end{example}


  \item For red and blue boxes, there are custom commands for titles, too:
    \begin{mathaside}[frametitle=\mathtitle{One Title}]
      Like this
    \end{mathaside}
    \begin{mathaside}[frametitle=\mathtitlep{Two Titles}{A Subtitle}]
      Or this
    \end{mathaside}
\end{itemize}

\hr{5in}

These styles also automatically apply to theorems and claims.

\begin{theorem}[Pythagorean Theorem]
  \label{thm:pyth}
  For any right triangle with legs $a,b$ and hypotenuse $c$:
  %
  \begin{equation}
    \label{eq:pyth}
    a^2+b^2=c^2
  \end{equation}
\end{theorem}
\begin{proof}
  This is left as an exercise to the reader.
\end{proof}

\begin{claim}
  This is the greatest note template in the world.
\end{claim}

\hr{5in}

There are different ways to quote things, too, depending on how you want to
emphasize:

\begin{quoting}
  This is a simple, indented quote with small letters and italics usually
  suitable for in-text quotations when you just want a block.
\end{quoting}

Alternatively, you can use the \smtt{\textbackslash inspiration} command from
the chapter heading, which leverages the \smtt{thickleftborder} frame
internally, but adds a little more padding and styling (there's also just
\smtt{leftborder} for a thinner variant):

\begin{thickleftborder}
  Hello there!
\end{thickleftborder}



\section{On Cross-Referencing}
\marginnote{\footnotesize\softtext This is the standard way to include margin
notes. There are also commands to link to source papers directly (see
\smtt{\textbackslash lesson}).} You can reference most things---see
\autoref{thm:pyth} or \eqref{eq:pyth} or the \nameref{ch:1} chapter---directly
and easily as long as you give them labels. These are ``built-ins.'' However,
you can also create a \term{custom term} that will be included in the index,
then include references to it that link back to the original definition. Try
clicking: \refterm{custom term}. Building the index is on you, though. You can
also reference by using a different term for the text: \reftermx{custom
term}{like this}. Sometimes it doesn't fit the \termx{grammar}{grammatical
structure} of the sentence so you can define the term one way and visualize it
another way (this creates a \aterm{grammar} entry in the index). There's also
\prop{math terms} and a way to reference them: \refprop{math terms} (clickable),
but they do \textbf{not} show up in the index.



\section{On Math}
Most of the math stuff is just macros for specific things like the convolution
operator, $\conv$, probabilities, $\cprob{A}{B=C}$, or big-$O$ notation,
$\bigO{n^2\log{n}}$ but there's also a convenient way to include explanations on
the side of an equation:
%
\begin{align*}
  1 + 1 &\overset{?}{=} 2    \sideblock{2in}{first we do this} \\
      2 &\overset{?}{=} 2    \sideblock{2in}{then we do this} \\
      2 &= 2 \qed
\end{align*}

These are all in the \smtt{CustomCommands.sty} file.



\end{document}

